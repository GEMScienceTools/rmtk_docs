This procedure performs the post-processing part of a set of dynamic analyses to first assemble a damage probability matrix, and then use this data for the derivation of a fragility function. Therefore the results of a set of dynamic analyses previously run have to be input to start the process. A list of intensity measure associated to each accelerogram, and corresponding EDP for each structure of the building class can be input, along with the set of limit states expressed in terms of the same EDP. The EDPs resulting from the dynamic analyses are compared with the limit state displacements and a global damage state is assigned to each structure. Thus, for each record, the number of frames in each damage state can be obtained. The distribution of buildings in each damage state is organised in the damage probability matrix, with a number of rows equal to the number of ground motion records and a number of columns equal to the number of damage states.

The processing of the data continue with the estimation of the cumulative fraction of structures in each damage state, summing the percentages of frames belonging to all the subsequent damage states. A lognormal cumulative distribution function, expressing the probability of exceeding each damage state in a continuous fashion, is then fit to these results, leading to the statistical parameters of the fragility curves. The regression analysis is carried out using the maximum likelihood method.

This function have the advantage of accounting for the record-to-record variability by the use of many ground motion records, and the inter-building variability subjecting to the same set of accelerograms hundreds of structures representing the entire building class.

To derive a discrete vulnerability function at certain intensity measure levels, the input damage-to-loss factors are applied to the probability of occurance of each damage state, extracted from the probability of exceedance of each damage state described by the fragility function. For the vector of selected intensity measure levels a value of loss ratio is thus defined.

\subsubsection{Options}
\label{subsubsec:NDMoptions}
In the Options the user has to define first of all the type of inputs at disposal, setting the variable \textit{in\_ type}, choosing between entering a damage count matrix, which corresponds to a damage probability matrix, where the probabilities of each damage state are substituted by the count of buildings in that damage state, or IML and corresponding EDPs for each dynamic analysis

\begin{Verbatim}[frame=single, commandchars=\\\{\}, samepage=true]
in\_type = 0 # damage count matrix 
in\_type = 1# IMLs and EDPs 
\end{Verbatim}

The variable \textit{vuln} instead gives the opportunity to decide the type of outputs, whether to stop the process at the derivation of the fragility curves, or to go all the way up to the vulnerability curve definition, applying damage-to-loss functions.

\begin{Verbatim}[frame=single, commandchars=\\\{\}, samepage=true]
vuln = 0 # derive fragility curves 
vuln = 1 # derive vulnerability curve
\end{Verbatim}

The variable g has to be set to the value of the gravity acceleration, expressed consistently with the units used for the intensity measure input data. For example if the intensity measure used is PGA expressed in g, the variable g will be set to 1, if PGA is expressed in m/s$^2$ instead g will be set to 9.81.

The variable \textit{iml} is a numpy array that identifies the intensity measure levels for which loss ratios are computed and provided in the vulnerability curve.

\begin{Verbatim}[frame=single, commandchars=\\\{\}, samepage=true]
iml = np.linspace(0.1,15,100)
\end{Verbatim}

The variable \textit{plotflag} allows or inhibits the displaying of plots. It is a python list composed of 2 integers, each one controlling a different plot: fragility and vulnerability function respectively. Each integer can take as value either zero or one, whether the corresponding graph has to be displayed or not:

\begin{Verbatim}[frame=single, commandchars=\\\{\}, samepage=true]
plotflag = [1, 1] # plot all the graphs
plotflag = [0, 0] # do not plot any graph
\end{Verbatim}

The following variables set some of the characteristics of the plots:
\begin{itemize}
\item IMlabel: list of one strings defining the IM on the x axis as ['Sa(Tel)-m/s$^{2}$']
\item linew: integer for defining lines width.
\item fontsize : fontsize used for labels, graphs etc.
\end{itemize}

\subsubsection{Inputs}
\label{subsubsec:NDMinputs}
The inputs must be formatted as comma-separated value files (.csv), and saved in the folder \textit{input}, contained in the NDP directory. If any other environment different from Windows is used make sure that the "comma separated values Windows" is selected as saving option when creating the input files.  

Two types of input can be entered, whether the results of the set of dynamic analyses performed have already been organised in a damage probability matrix for or not. In the former case the variable \textit{in\_type} should be set to 0 and the damage count matrix should be input in the csv file \textit{dcm.csv}. The first two columns refer to the number of record and the corresponding intensity measure level, the following columns report the number of buildings in each damage state, as shown in the following table:

\begin{table}[!htbp]
\centering
\begin{tabular}{|c|c|c|c|c|} \hline
\textbf{n.records} & \textbf{Intensity Measure Level} & \textbf{DS$_0$} & \textbf{DS$_1$} & \textbf{DS$_2$} \\ \hline
1 & 49.852 &	30 &	16 &	54\\ \hline
2 & 47.056 &	54 &	15 &	31\\ \hline
3 & 33.012 &	59 &	10 &	31\\ \hline
4 & 82.125 &	24 &	26 &	50\\ \hline
... & ... & ... & ... & ... \\ \hline
5 & 37.499 &	58 &	5 &	37\\ \hline
\end{tabular}
\end{table}

In the latter case the variable \textit{in\_type} should be set to 1, and the results of the set of dynamic analyses should be entered in the \textit{edp.csv} file in the following fashion: number of record, corresponding IML, and corresponding EDPs for each building subjected to that record.

\begin{table}[!htbp]
\centering
\begin{tabular}{|c|c|c|c|c|c|} \hline
\textbf{n.records} & \textbf{IML} & \textbf{edp$_{blg,1}$} & \textbf{edp$_{blg,2}$} & \textbf{...} & \textbf{edp$_{blg,n}$} \\ \hline
1 &	69.209 &	-0.00069 &	0.00031 & ... &	0.00131\\ \hline
2 &	75.470 &	0.00102 & 	0.00202 & ... &	0.00302\\ \hline
3 &	62.233 &	-0.00090 &	0.00010 & ... &	0.00110\\ \hline
4 &	168.47 &	-0.00246 &	-0.00146 & ... &	-0.00046\\ \hline
5 &	67.612 &	0.00095 & 	0.00195 & ... &	0.00295\\ \hline
... & ... & ... & ... & ... & ...\\ \hline
n &	34.484 &	0.00036 & 	0.00136 & ... & 	0.00236\\ \hline
\end{tabular}
\end{table}

In this case also the limit states must be input in the \textit{limits.csv} file. Each line of the file corresponds to the limit states of a building, but if all the buildings share the same limits a single line can be input.

\begin{table}[!htbp]
\centering
\begin{tabular}{|c|c|c|c|c|} \hline
\textbf{n.building} & \textbf{LS$_1$} & \textbf{LS$_2$} & \textbf{LS$_3$} & \textbf{LS$_4$} \\ \hline
1 & 0.003 &	0.010 &	0.025 &	0.0337\\ \hline
2 & 0.004 &	0.015 &	0.020 &	0.035\\ \hline
3 & 0.002 &	0.019 &	0.027 &	0.032\\ \hline
... & ... & ... & ... & ...\\ \hline
n & 0.0024 &	0.016 &	0.025 &	0.03\\ \hline
\end{tabular}
\end{table}

\subsubsection{Calculations}
The overall workflow of the DPM-based procedure is summarised in this section. The option \textit{in\_type} should be set according to the input at disposal. The corresponding inputs should follow the requirements described in the previous section. At this point the code proceeds with the following steps:

\begin{enumerate}
\item In the function \textit{read\_data} the inputs are read and the damage count matrix is returned.
\begin{enumerate}
\item If \textit{in\_type} = 0, the damage count matrix is extracted directly from the \textit{dcm.csv} file.

\item If \textit{in\_type} = 1 the IMLs of the records used in the dynamic analyses and the corresponding EDPs are extracted from \textit{edp.csv} and the limit states for each building, expressed in terms of the same EDP, are extracted from \textit{limits.csv}.	 These data are converted into a damage count matrix according to the method described in section \ref{subsec:nld-dpm}.
\end{enumerate}

\item The parameters extracted are used to derive the Probability of Exceedance (PoE) of each limit state for each IML, as described in section \ref{subsec:nld-dpm}.

\item The PoEs are fitted with a lognormal function using the maximum likelihood method. The mean $\mu_{ln IML}$ and standard deviation $\sigma_{ln IML}$ of the corresponding normal distribution for the entire class of buildings are found for each limit state.

\item The $\mu_{ln IML}$ and $\sigma_{ln IML}$ are converted to logarithmic $\mu_{IML}$ and $\sigma_{IML}$. The fragility curves for the class of buildings are displayed if the variable \textit{plotflag}[0] = 1, and the logarithmic $\mu_{IML}$ and $\sigma_{IML}$ are exported in the \textit{outputs} folder.

\item If \textit{vuln} =1:  For the intensity measure levels defined in the variable \textit{iml} a value of loss ratio is defined, according to section \ref{subsec:nld-dpm}. A vulnerability curve for the class of buildings is displayed if the variable~\textit{plotflag}[1] = 1, and the loss ratios at each iml are exported in the \textit{outputs} folder.

\end{enumerate}