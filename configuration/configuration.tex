%%%%%%%%%%%%%%%%%%%%%%%%%%%%%%%%%%%%%%%%%
% The Legrand Orange Book
% LaTeX Template
% Version 1.4 (12/4/14)
%
% This template has been downloaded from:
% http://www.LaTeXTemplates.com
%
% Original author:
% Mathias Legrand (legrand.mathias@gmail.com)
%
% License:
% CC BY-NC-SA 3.0 (http://creativecommons.org/licenses/by-nc-sa/3.0/)
%
% Compiling this template:
% This template uses biber for its bibliography and makeindex for its index.
% When you first open the template, compile it from the command line with the 
% commands below to make sure your LaTeX distribution is configured correctly:
%
% 1) pdflatex rmtk-docs
% 1a) pdflatex -shell-escape rmtk-docs
% 2) makeindex rmtk-docs.idx -s StyleInd.ist
% 3) biber rmtk-docs
% 4  makeglossaries rmtk-docs
% 4) pdflatex rmtk-docs x 2
%
% After this, when you wish to update the bibliography/index use the appropriate
% command above and make sure to compile with pdflatex several times 
% afterwards to propagate your changes to the document.
%
% This template also uses a number of packages which may need to be
% updated to the newest versions for the template to compile. It is strongly
% recommended you update your LaTeX distribution if you have any
% compilation errors.
%
% Important note:
% Chapter heading images should have a 2:1 width:height ratio,
% e.g. 920px width and 460px height.
%
%%%%%%%%%%%%%%%%%%%%%%%%%%%%%%%%%%%%%%%%%

%----------------------------------------------------------------------------------------
%	PACKAGES AND OTHER DOCUMENT CONFIGURATIONS
%----------------------------------------------------------------------------------------

\documentclass[11pt,fleqn]{book} % Default font size and left-justified equations

\usepackage[top=3cm,bottom=3cm,left=3.2cm,right=3.2cm,headsep=10pt,a4paper]{geometry} % Page margins

\usepackage{xcolor} % Required for specifying colors by name
\definecolor{ocre}{RGB}{243,102,25} % Define the orange color used for highlighting throughout the book

% Font Settings
\usepackage{avant} % Use the Avantgarde font for headings
%\usepackage{times} % Use the Times font for headings
\usepackage{mathptmx} % Use the Adobe Times Roman as the default text font together with math symbols from the Sym­bol, Chancery and Com­puter Modern fonts

\usepackage{microtype} % Slightly tweak font spacing for aesthetics
\usepackage[utf8]{inputenc} % Required for including letters with accents
\usepackage[T1]{fontenc} % Use 8-bit encoding that has 256 glyphs

% Bibliography
\usepackage{csquotes}
\usepackage[style=alphabetic,
            sorting=nyt,
            sortcites=true,
            natbib=true,
            style=authoryear,
            maxcitenames=2,
            maxbibnames=100,
            autopunct=true,
            babel=hyphen,
            hyperref=true,
            doi=true,
            abbreviate=false,
            backref=true,
            backend=bibtex,
	    	uniquename=false,
	    	uniquelist=false]{biblatex}
\addbibresource{./bibliography/rmtk.bib} % BibTeX bibliography file
\defbibheading{bibempty}{}

% Figure caption settings
\usepackage[textfont=it,margin=10pt,font=small,labelfont=bf,labelsep=endash]{caption}
\usepackage{subcaption}
\usepackage{rotating}

% Table - colors from
\usepackage{verbatim}
\usepackage{color, colortbl}
\definecolor{almond}{rgb}{0.94, 0.87, 0.8}
\definecolor{ashgrey}{rgb}{0.7, 0.75, 0.71}
\definecolor{anti-flashwhite}{rgb}{0.95, 0.95, 0.96}
\definecolor{airforceblue}{rgb}{0.36, 0.54, 0.66}

% Index
\usepackage{calc} % For simpler calculation - used for spacing the index letter headings correctly
\usepackage{makeidx} % Required to make an index
% \setcounter{tocdepth}{3}    % entries down to \subsubsections in the TOC
\makeindex % Tells LaTeX to create the files required for indexing

\usepackage{todonotes}
\usepackage{geometry}
\usepackage{marginnote}

%
% Package to create a glossary - It must be uploaded after hyperref
% to produce the glossary: makeglossaries OQB
\usepackage[acronym,nonumberlist,style=altlist]{glossaries}
\glstoctrue
\makeglossaries

% package for bold symbols
\usepackage{bm}

% for better looking tables
\usepackage{ctable}
\usepackage{microtype}

% for listing Python code
\usepackage{listings}

%
%----------------------------------------------------------------------------------------
% Trees
%\usepackage[pdf]{pstricks}
%\usepackage{auto-pst-pdf}
%\usepackage{pst-tree}
%