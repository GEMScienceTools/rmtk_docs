These fragility functions can be used directly by the Scenario Damage or the Classiclal PSHA-based Damage calculators of the OpenQuake-engine (\cite{SilvaEtAl2014a}; \cite{PaganiEtAl2014a}). \\

In order to use the results of the nonlinear static procedures (see Section \ref{sec:record-nsp}) for the derivation of a fragility model, there are a number of attributes that need to be specified. Each function must be related with a building class, which must be defined using the parameter \verb=taxonomy=. Additional information about the GEM building taxonomy can be found in \cite{BrzevEtAl2013}, and a tool to compile the GEM taxonomy can be found on the OpenQuake-platform (https://taxtweb.openquake.org/).\\

For what concerns the definition of the seismic input, it is necessary to establish the intensity measure type that should be considered using the variable \verb=IMT=. Currently, the Risk Modeller's Toolkit supports \verb=PGA=, \verb=PGV= and \verb=Sa=. If the latter intensity measure type is chosen, it is also necessary to establish the period of vibration (in seconds) using the variable \verb=T=, and the elastic damping using the variable \verb=damping=. Finally, the range of applicability of the fragility model should be defined using the \verb=minIML= and \verb=maxIML=.  \\

The results in the probability damage matrix (\verb=PDM=) are used to fit a lognormal cumulative function, with a logarithmic mean ($\mu$) and logarithmic standard deviation ($\sigma$). These two parameters are calculated using one of the two currently implemented statistical methods: least squares or the maximum likelihood. The former approach estimates a solution ($\mu$, $\sigma$) that minimizes the sum of the squares of the errors (i.e. difference between the prediction of the lognormal function and the data). The latter method leads to a solution that maximizes the likelihood function. A comprehensive description of the strengths and limitations of these methodologies can be found in \cite{LallemantEtAl2015}. The method that should be followed must be specified by setting the parameter \verb=regression_method= to \verb=least squares= or to \verb=maximum likelihood=.\\

The calculation of the fragility model also requires the set of ground motion records used in the analytical analyses (\verb=gmrs=), and the damage model utilized to allocated each structure into a damage state (\verb=damage_model=). A description of these two components have been provided in Section \ref{subsec:gmrs} and \ref{subsec:dmg_model}, respectively.\\

The function that calculates fragility models is contained in the module \verb=utils=. An example of this process is depicted below.

\begin{Verbatim}[frame=single, commandchars=\\\{\}, samepage=true]
IMT = 'Sa'
T = 0.3
regression_method = 'least squares'
taxonomy = 'RC'
minIML = 0.01
maxIML = 1
fragility_model = utils.calculate_mean_fragility(gmrs,PDM,T,damping,...
IMT,damage_model,regression_method)
\end{Verbatim}

Once the parameters ($\mu$, $\sigma$) of the fragility model have been calculated, it is possible to save these results using the function \verb=save_mean_fragility=. This feature can export the fragility model using the OpenQuake-engine format (\verb=nrml=), or following a \verb=csv= format. This indication should be defined using the variable \verb=output_type=. It is also possible to create a plot of the resulting model, using the function \verb=plot_fragility_model=. In order to use these functions, it is necessary to import the module \verb=utils=. This process is demonstrated below. 

\begin{Verbatim}[frame=single, commandchars=\\\{\}, samepage=true]
output_type = 'nrml'
utils.save_mean_fragility(taxonomy,fragility_model,minIML,maxIML,...
output_type)
utils.plot_fragility_model(fragility_model,minIML,maxIML)
\end{Verbatim}

A detailed description of the \verb=nrml= format can be found on the OpenQuake-engine manual \citep{GEM2015}. For what concerns the structure of the \verb=csv= format, an example is provided in Table \ref{table:ff_csv}. The first row contains the building \verb=taxonomy=, the intensity measure type (\verb=IMT=), the minimum and maximum intensity measure levels (\verb=minIML=, \verb=maxIML=). The second row comprises the titles of the information stored in each column: damage states; logarithmic mean; logarithmic standard deviation; mean; standard deviation; median and coefficient of variation. The remaining columns contain the results for each damage state.

\begin {table}[htb]
\caption{Example of a fragility model stored following a csv format.}
\label{table:ff_csv}
\begin{center}
  \begin{tabular}{ | c | c | c | c | c | c | c |}
  \hline
RC & Sa(0.3) & 0.01 & 1.0 &  &  &  \\ \hline
Damage state & log mean & log stddev & mean & stddev & median & cov \\ \hline
Slight & -2.67 & 0.28 & 0.07 & 0.02 & 0.07 & 0.29 \\ \hline
Moderate & -2.37 & 0.30 & 0.10 & 0.03 & 0.09 & 0.31 \\ \hline
Extensive & -0.26 & 0.86 & 1.12 & 1.16 & 0.77 & 1.04 \\ \hline
Collapse & 0.42 & 0.96 & 2.42 & 2.99 & 1.52 & 1.24 \\ \hline
  \end{tabular}
\end{center}
\end{table}

Finally, a folder containing a set of fragility functions for buildings of different typologies derived using the RMTK and saved using the CSV format can be used to create a fragility model for use in OpenQuake risk analyses. In order to use the function \verb=save_fragility_set_nrml=, it is necessary to import the module \verb=utils=. The path to the folder containing the individual CSV fragility files, and the name of the destination XML file are the required inputs for this function. Usage of this function is shown below:

\begin{Verbatim}[frame=single, commandchars=\\\{\}, samepage=true]
utils.save_fragility_set_nrml(folder, destination_file)
\end{Verbatim}