These vulnerability functions can be used directly by the Scenario Risk, Classical PSHA-based Risk and Probabilistic Event-based Risk calculators of the OpenQuake-engine (\cite{SilvaEtAl2014a}; \cite{PaganiEtAl2014a}). \\

A vulnerability model can be derived directly from loss data (either analytically generated or based on past seismic events), or by combining a set of fragility functions with a consequence model (see Section \ref{subsec:cons_model}). In this process, the fractions of buildings in each damage state are multiplied by the associated damage ratio (from the consequence model), in order to obtain a distribution of loss ratio for each intensity measure type. Currently only the latter approach is implemented in the Risk Modeller's Toolkit, though the former method will be included in a future release.\\

The location of the consequence model must be defined using the parameter \verb=cons_model_file=, and loaded into the Risk Modeller's Toolkit using the function \verb=read_consequence_model=. The intensity measure levels for which the distribution of loss ratio will be calculated must be defined using the variable \verb=imls=.\\

The Risk Modeller's Toolkit allows the propagation of the uncertainty in the consequence model to the vulnerability function. Thus, instead of just providing a single loss ratio per intensity measure type, it is possible to define a probabilistic model (following a \verb=lognormal= or \verb=beta= functions) or a non-parametric model (i.e. probability mass function - \verb=PMF=). This model must be defined using the variable \verb= distribution_type=.\\

The derivation of the vulnerability function also requires the previously computed \verb=fragility_model=. The function that calculates this result is contained in the module \verb=utils=. An example of this process is depicted below.

\begin{Verbatim}[frame=single, commandchars=\\\{\}, samepage=true]
cons_model_file = '../../../../../../rmtk_data/cons_model.csv'
cons_model = utils.read_consequence_model(cons_model_file)
imls = [0.1,0.2,0.3,0.4,0.5,0.6,0.8,0.9,1.0]
distribution_type = 'PMF'
vul_model = utils.convert_fragility_vulnerability(fragility_model,...
cons_model,imls,type_distribution)
\end{Verbatim}

The resulting vulnerability function can be saved using the function \verb=save_vulnerability=. This feature can export the vulnerability function using the OpenQuake-engine format (\verb=nrml=), or following a \verb=csv= format. Similarly to what was described for the fragility models, this indication should be provided using the variable \verb=output_type=. It is also possible to plot vulnerability functions, using the function \verb=plot_vulnerability_model=. In order to use these functions, it is necessary to import the module \verb=utils=. This process is demonstrated below.

\begin{Verbatim}[frame=single, commandchars=\\\{\}, samepage=true]
output_type = 'nrml'
utils.save_vulnerability(taxonomy,vulnerability_model,output_type)
utils.plot_vulnerability_model(vulnerability_model)
\end{Verbatim}

A detailed description of the \verb=nrml= format for vulnerability functions can be found on the OpenQuake-engine manual \citep{GEM2015}. For what concerns the structure of the \verb=csv= file, this format varies depending on how the uncertainty is being defined: parametric (lognormal or beta) or non-parametric (probability mass function). For the former case, an example is provided in Table \ref{table:vf_cont_csv}. The first row contains the building \verb=taxonomy=, the intensity measure type (\verb=IMT=), and the type of probabilistic model used to represent the uncertainty. The second row comprises the list of the intensity measure levels, and the means and associated coefficients of variation are provided in the third and fourth row, respectively.

\begin {table}[htb]
\caption{Example of a vulnerability model with a parametric uncertainty modelling.}
\label{table:vf_cont_csv}
\begin{center}
  \begin{tabular}{ | c | c | c | c | c | c | c | c | c | c |}
  \hline
RC & Sa(0.3) & lognormal &  &  &  &  &  &  & \\ \hline
imls & 0.10 & 0.20 & 0.30 & 0.40 & 0.50 & 0.60 & 0.80 & 0.90 & 1.00\\ \hline
mean & 0.00 & 0.02 & 0.05 & 0.08 & 0.12 & 0.17 & 0.25 & 0.29 & 0.33\\ \hline
cov & 0.36 & 0.23 & 0.18 & 0.13 & 0.07 & 0.04 & 0.02 & 0.02 & 0.00\\ \hline
  \end{tabular}
\end{center}
\end{table}

For what concerns the \verb=csv= format for vulnerability functions using the non-parametric approach, an example can be found in Table \ref{table:vf_pmf_csv}. The first two rows are similar to the previous case, and the remaining columns contain the probability of having a given loss ratio, conditional on an intensity measure level.

\begin {table}[htb]
\caption{Example of a vulnerability model with a non-parametric uncertainty modelling.}
\label{table:vf_pmf_csv}
\begin{center}
  \begin{tabular}{ | c | c | c | c | c | c | c | c | c | c |}
  \hline
RC & Sa(0.3) & PMF &  &  &  &  &  &  & \\ \hline
imls & 0.10 & 0.20 & 0.30 & 0.40 & 0.50 & 0.60 & 0.80 & 0.90 & 1.00\\ \hline
loss ratio &  \multicolumn{9}{| c |}{probabilities} \\ \hline
0.00 & 0.80 & 0.15 & 0.00 & 0.00 & 0.00 & 0.00 & 0.00 & 0.00 & 0.00\\ \hline
0.11 & 0.20 & 0.60 & 0.30 & 0.00 & 0.00 & 0.00 & 0.00 & 0.00 & 0.00\\ \hline
0.22 & 0.00 & 0.25 & 0.60 & 0.40 & 0.00 & 0.00 & 0.00 & 0.00 & 0.00\\ \hline
0.33 & 0.00 & 0.00 & 0.10 & 0.50 & 0.20 & 0.00 & 0.00 & 0.00 & 0.00\\ \hline
0.44 & 0.00 & 0.00 & 0.00 & 0.10 & 0.70 & 0.10 & 0.00 & 0.00 & 0.00\\ \hline
0.56 & 0.00 & 0.00 & 0.00 & 0.00 & 0.10 & 0.50 & 0.10 & 0.00 & 0.00\\ \hline
0.67 & 0.00 & 0.00 & 0.00 & 0.00 & 0.00 & 0.35 & 0.30 & 0.00 & 0.00\\ \hline
0.78 & 0.00 & 0.00 & 0.00 & 0.00 & 0.00 & 0.05 & 0.50 & 0.10 & 0.00\\ \hline
0.89 & 0.00 & 0.00 & 0.00 & 0.00 & 0.00 & 0.00 & 0.10 & 0.70 & 0.00\\ \hline
1.00 & 0.00 & 0.00 & 0.00 & 0.00 & 0.00 & 0.00 & 0.00 & 0.20 & 1.00\\ \hline
  \end{tabular}
\end{center}
\end{table}

Finally, a folder containing a set of vulnerability functions for buildings of different typologies derived using the RMTK and saved using the CSV format can be used to create a vulnerability model for use in OpenQuake risk analyses. In order to use the function \verb=save_vulnerability_set_nrml=, it is necessary to import the module \verb=utils=. The path to the folder containing the individual CSV vulnerability files, and the name of the destination XML file are the required inputs for this function. Usage of this function is shown below:

\begin{Verbatim}[frame=single, commandchars=\\\{\}, samepage=true]
utils.save_vulnerability_set_nrml(folder, destination_file)
\end{Verbatim}