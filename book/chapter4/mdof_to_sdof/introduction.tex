% A conventional pushover curve describes the relation between base shear and roof displacement of a MDOF structure when an increasing lateral force is applied. The use of pushover curves in earthquake engineering somewhat originates from the pioneering work or Gulkan and Sozen [36], in which simplified SDOF models were created to represent MDOF structures and used in nonlinear static analysis. This methodology has many advantages and disadvantages that have been the focus of several studies for the past years, in particular that by Krawinkler and Seneviratna [37]. The latter stated that such approach is a valuable tool in vulnerability assessment because of its simplicity, ease of use, and reduced running time, despite its inability to reproduce certain phenomena such as viscous damping, strength deterioration or pinching effect. These authors also highlighted the constant loading pattern as one of the weakest points of this method, as it ignores some deformation modes that are propelled by dynamic response and inelastic response characteristics. This invariant loading pattern usually adopts a uniform, triangular or a first deformation mode shape. In this study, the first two patterns were considered but not the latter, because the regularity of the RC frames led to a first deformation mode approximately of a triangular shape, thus leading to the same structural behavior. Instead, a decision was taken to apply a modal loading pattern with the resulting shape from the contribution of the first three modes of vibration.
% The transformation of the pushover curve from the MDOF system to a capacity curve in terms of spectral acceleration (Sa) versus spectral displacement (Sd) for an equivalent SDOF structure can be carried out in various ways, under the condition that the deformed shape of the structure is not significantly altered during the dynamic loading. The roof displacement has been converted to Sd herein on the basis of the participation factor of the first mode of vibration, whereas the base shear has been reduced to Sa using the same factor and the first modal mass.

Several structural analysis packages allow the user to perform a reliable pushover analysis on a nonlinear model of the MDOF structure. Often, these MDOF pushover cuves need to be converted to simplified SDOF models for use in nonlinear static analysis methods. This module allows the conversion of the MDOF results into 'equivalent SDOF' results, thus making them compatible with a wide range of non-linear static procedures. At present, two methods are provided in this module for obtaining equivalent SDOF capacity curves for an MDOF system.

% A pushover curve is generated by subjecting a detailed or a carefully simplified structural model to one or more lateral load patterns and then increasing the magnitude of the total load to generate a nonlinear inelastic force-deformation relationship. The load vector is usually a representation of the load acting on the structure vibrating in its first mode .In ATC-40 Procedure A , the global parameters used are spectral acceleration and spectral displacement. Therefore a Capacity curve used in this procedure is a curve obtained by transforming the structure base shear vs. roof displacement curve into an Equivalent Single Degree of Freedom structure, acceleration vs. displacement curve .In this chapter equations needed to convert the pushover curve into a capacity curve will be developed.