This feature of the Risk Modeller's Toolkit enables the user to derive equivalent single-degree-of-freedom (SDOF) capacity curves for multiple-degree-of-freedom (MDOF) structures. Several structural analysis packages allow the user to perform a reliable pushover analysis on a nonlinear model of the MDOF structure. The results of the pushover analysis (in terms of a pushover curve) can be provided as input to one of the methods in this module. This module allows the conversion of the MDOF results into 'equivalent SDOF' results, thus making them compatible with a wide range of non-linear static procedures. At present, two methods are provided in this module for obtaining equivalent SDOF capacity curves for an MDOF system.

A pushover curve is generated by subjecting a detailed or a carefully simplified structural model to one or more lateral load patterns and then increasing the magnitude of the total load to generate a nonlinear inelastic force-deformation relationship. The load vector is usually a representation of the load acting on the structure vibrating in its first mode .In ATC-40 Procedure A , the global parameters used are spectral acceleration and spectral displacement. Therefore a Capacity curve used in this procedure is a curve obtained by transforming the structure base shear vs. roof displacement curve into an Equivalent Single Degree of Freedom structure, acceleration vs. displacement curve .The equivalent SDOF properties correspond to the first mode properties of the detailed structure .In this chapter equations needed to convert the pushover curve into a capacity curve will be developed.