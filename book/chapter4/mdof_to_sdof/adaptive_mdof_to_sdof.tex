Adaptive pushover methods have the advantage of better accounting for stiffness degradation, influence of higher mode effects, and spectral amplifications because of ground motion frequency content. The method used for the determination of the ‘equivalent SDOF adaptive capacity curve’ in this module is the approach recommended in \citet{Casarotti2007a}. In this method, the equivalent SDOF capacity curve is calculated step by step based on the actual displacements, rather than a transformation of the capacity curve referred to the roof displacement. Instead of relying on an invariant elastic or inelastic modal shape, the equivalent system displacement and acceleration are calculated at each analysis step, using the actual displacements at that step.

The equivalent system modal participation factor $\Gamma_{sys, k}$ at step $k$ is calculated as:
\begin{equation}
	\Gamma_{sys,k} = \frac{\sideset{}{_i}\sum m_i \Delta_{i,k}}{\sideset{}{_i}\sum m_i \Delta_{i,k}^2}
\end{equation}

The equivalent system displacement $\Delta_{sys,k}$ at step $k$ is calculated as:
\begin{equation}
	\Delta_{sys,k} = \frac{\sideset{}{_i}\sum m_i \Delta_{i,k}^2}{\sideset{}{_i}\sum m_i \Delta_{i,k}}
\end{equation}

Note that $\Delta_{sys,k}$ is defined to be the inverse of the modal participation factor.

The equivalent system mass $M_{sys,k}$ at step $k$ is defined as:
\begin{equation}
	M_{sys,k} = \frac{\sideset{}{_i}\sum m_i \Delta_{i,k}}{\Delta_{sys,k}}
\end{equation}

The equivalent system acceleration $S_{a-capacity}$ and displacement $S_{d-capacity}$ are calculated as:
\begin{equation}
	S_{a-capacity} = \frac{V_{b-pushover}}{M_{sys} g}
\end{equation}

\begin{equation}
	S_{d-capacity} = \frac{1}{\Gamma_{sys}}
\end{equation}