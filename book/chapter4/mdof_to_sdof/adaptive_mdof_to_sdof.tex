Adaptive pushover methods have the advantage of better accounting for stiffness degradation, influence of higher mode effects, and spectral amplifications because of ground motion frequency content. In this method, instead of applying an invariant load vector, the structural properties of the model are evaluated at each step of the analysis, and the loading pattern is updated accordingly. In this way, the variation in the structural stiffness at different deformation levels, and consequently the system degradation and period elongation can be accounted for. The only apparent drawback of this methodology can be the additional computation time required to assess the structural characteristics at every step.

The equivalent system displacement $\Delta_{sys,k}$ at step $k$ is calculated as:
\begin{equation}
	\Delta_{sys,k} = \frac{\sideset{}{_i}\sum m_i \Delta_{i,k}^2}{\sideset{}{_i}\sum m_i \Delta_{i,k}}
\end{equation}
Note that $\Delta_{sys,k}$ is defined to be the inverse of the modal participation factor.

The equivalent system acceleration $\S_{a-cap,k}$ at step $k$ is calculated as:
\begin{equation}
	S_{a-cap,k} = \frac{V_{b,k}}{M_{sys,k} g}
\end{equation}

The equivalent system mass $M_{sys,k}$ is defined as:
\begin{equation}
	M_{sys,k} = \frac{\sideset{}{_i}\sum m_i \Delta_{i,k}}{M_{sys,k} g}
\end{equation}
Note that $M_{sys,k}$ is the modal mass of the system at analysis step $k$.