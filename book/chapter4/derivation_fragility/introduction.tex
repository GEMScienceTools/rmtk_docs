As explained by Lin and Miranda (2008), performance-based evaluation of structures requires the estimation of a demand spectrum that can properly take into account the inelastic behaviour of the structure(s) under analysis, namely, their capacity to dissipate seismic energy through hysteresis (Monteiro, 2011). In order to achieve this, two main approaches exist. The first one estimates the maximum inelastic deformation of a structure by applying a modification factor to its maximum linear elastic deformation. This factor can be damping-based or ductility-based. The former case reduces both the displacement and acceleration spectral coordinates by a reduction factor B, which is based on the equivalent viscous damping of the system. On the other hand, the latter reduce the spectral acceleration ordinates of a 5\%-damped elastic response spectrum by a factor dependant on the ductility of the system.  

The second approach estimates the maximum deformation of a nonlinear structure as the maximum deformation of an equivalent linear system with longer period of vibration (i.e. more flexible lateral stiffness) and higher viscous damping than those of the “original” structure. These parameters can be dependant either on the displacement ductility of the system (which implies an iterative process) or on the strength ratio R (or strength reduction factor). 