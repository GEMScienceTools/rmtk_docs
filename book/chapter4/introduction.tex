% Vulnerability functions, a fundamental component in the process of assessing seismic risk, can be defined as the probabilistic distribution of loss ratio conditional on a certain level of ground motion. Fragility functions, defining the probability of exceeding a set of damage states, can be combined with a consequence model, which establishes the relation between physical damage and a percentage of loss, to derive vulnerability functions. Building damage and repair cost data from past earthquakes can be used to derive both of these types of models [1–3]. However, empirical methodologies can have some disadvantages such as the subjectivity in allocating each building to a damage state or the lack of accuracy in the determination of the ground motion affecting the region. Furthermore, there are only a few dozen places in the world where post-earthquake damage and repair cost data has been collected from a number of buildings large enough to permit the development of reliable vulnerability functions. To overcome these limitations, analytical methodologies can be employed in either a single structure that is believed to be representative of a class of buildings, or a set of randomly generated buildings, modelled using structural analysis techniques, and subjected to specific lateral loading patterns or accelerograms.

% The various analytical methodologies for structural assessment can be categorized in two main groups: nonlinear dynamic analysis and nonlinear static analysis, each one having its own strengths and shortcomings. The main advantage in employing nonlinear dynamic analysis is certainly the fact that the actual dynamic phenomenon is reproduced by applying an acceleration time history at the base of the structure, leading in theory to more accurate results. However, the intrinsic modelling complexity (e.g., hysteric response models, equivalent viscous damping) combined with the heavy computational effort, is often impractical, thus favoring the employment of simpler methods, comprising nonlinear static analysis [13]. In this second approach, pushover curves are computed and used to estimate the maximum displacement response experienced by the structure for a given ground motion record. The main drawback of this simplified methodology lies with the assumption that the structural response obtained from horizontal static loading is representative of the one attained in the dynamic analysis.

% In this paper, several analytical methodologies are used to derive fragility functions for the same structural typology. A number of static approaches are investigated herein based on conventional and adaptive pushover analyses together with nonlinear static procedures (e.g., capacity spectrum method (CSM) [14], displacement coefficient method (DCM) [15], N2 method [16]), by using hundreds of ground motion records, to derive fragility functions for different levels of ground motion (intensity measure levels). Then, dynamic analysis is used as the baseline method in this sensitivity study, to yield conclusions regarding the relative accuracy of each method. Each set of fragility functions is transformed into vulnerability functions (i.e., probability of loss for a given level of ground motion) by calculating the mean damage ratio (i.e., ratio of cost of repair to cost of replacement) for a number of intensity measure levels

% The so-called NSP represent a simplified approach for the assessment of the seismic behavior of structures, included in guidelines such as the ATC-40 [14] and FEMA-440 [15] in the United States or the Eurocode 8 [43] in Europe. In this study, four distinct methodologies were employed: the CSM [14], the coefficient displacement method [15], the N2 method [16], and the adaptive CSM (ACSM) [44], which are further described in the following sections. These methodologies make use of capacity curves in terms of Sa versus Sd (i.e., the capacity of the equivalent SDOF). Each NSP is employed to estimate the target displacement obtained for each ground motion record, and this level of displacement is used to allocate the building in a damage state (according to the limit state criteria define in Section 2). This target displacement can be equated to the maximum roof displacement that would be experienced by the equivalent SDOF structure in a nonlinear dynamic analysis.

% Other NSPs such as the modal pushover analysis [45] or the adaptive modal combination procedure [46] have not yet been considered.