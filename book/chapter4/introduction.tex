Seismic fragility and vulnerability functions form an integral part of a seismic risk assessment project, along with the seismic hazard and exposure models. Fragility functions for a building or a class of buildings are typically associated with a set of discrete damage states. A fragility function defines the probabilities of exceedance for each of these damage states as a function of the intensity of ground motion. A vulnerability function for a building or a class of buildings defines the probability of exceedance of loss values as a function of the intensity of ground motion. A consequence model, sometimes also referred to as a damage-to-loss model, which describes the expected loss for different damage states, can be used to derive the vulnerability function for a building or a class of buildings from the corresponding fragility function.

Empirical methods are often preferred for the derivation of fragility and vulnerability functions when relevant data regarding the levels of physical damage and loss at various levels of ground shaking are available from past earthquakes. However, the major drawback of empirical methods is the highly limited quantity and quality of damage and repair cost data and the corresponding ground shaking intensities available from previous events.

The analytical approach to derive fragility and vulnerability functions for an individual structure relies on creating a numerical model of the structure and assessing the deformation behaviour of the modelled structure by subjecting it to selected ground motion acceleration records or predetermined lateral load patterns. The deformation then needs to be related to physical damage to obtain the fragility functions. The fragility functions can be combined with the appropriate consequence model to derive the vulnerability function for the structure. Fragility and vulnerability functions for a class of buildings (a "building typology") can be obtained by considering a number of structures considered representative of that class. A combination of Monte Carlo sampling followed by regression analysis can be used to obtain a single "representative" fragility or vulnerability function for the building typology.

The level of sophistication employed during the structural analysis stage is constrained both by the amount of time and the types of information regarding the structure available to the modeller. Although performing nonlinear dynamic analysis of a highly detailed model of the structure using several accelerograms is likely to yield a more representative picture of the dynamic deformation behaviour of the real structure during earthquakes, nonlinear static analysis is often preferred due to the lower modelling complexity and computational effort demanded by static methods. Different researchers have proposed different methodologies to derive fragility functions using pushover or capacity curves from nonlinear static analyses. Several of these methodologies have already been implemented in the RMTK. The following sections of this chapter describe some of these techniques in more detail.