Ruiz-Garcia and Miranda (2007), Vamvatsikos and Cornell (2006) and Dolsek and Fajfar (2004) studies on the assessment of nonlinear structural response, have been integrated in three nonlinear static procedures, which are based on the use of capacity curves, resulting from nonlinear static pushover analysis, to determine directly the median seismic intensity values $\hat{S}_a$ corresponding to the attainment of a certain damage state threshold (limit state) and the corresponding dispersion $\beta_{S_a}$. These parameters are used to represent a fragility curve as the probability of the limit state capacity C being exceeded by the demand D, both expressed in terms of intensity levels (S$_a, ds$ and $S_a$ respectively), as shown in the following equation:

\begin{equation}
P_{LS}(S_a) = P(C < D | S_a) = \Phi(\frac{ln S_a -ln \hat{S}_a, ds}{\beta_{S_a}})
\label{eq:fragility-definition}
\end{equation}

The methodologies implemented allow to consider different shapes of the pushover curve, multilinear and bilinear, record-to-record dispersion and dispersion in the damage state thresholds in a systematic and harmonised way. 

The intensity measure to be used is S$_a$ and a mapping between any engineering demand parameter (EDP), assumed to describe the damage state thresholds, and the roof displacement should be available from the pushover analysis.

The methodologies are originally built for single building fragility curves, however the fragility curves derived for single buildings can be combined in a unique fragility curve, which considers also the inter-building uncertainty, as described in the Section \ref{subsec:SPO2IDA}.