This procedure aims to determine the displacements from an inelastic spectra for systems with a given ductility factor. The inelastic displacement spectra is determined by means of applying a ductility-based reduction factor (C), which depends on the natural period of the system, the given ductility factor, the hysteretic behaviour, the damping model, and the frequency content of the ground motion.\\

The procedure proposed by \citep{VidicEtAl1994} was validated by a comparison of the approximate spectra with the “exact” spectra obtained from non-linear dynamic time history analyses. Records from California and Montenegro were used as representative of “standard” ground motion, while the influence of input motion was analysed using other five groups of records (coming from different parts of the world) that represented different types of ground motions. The influence of the hysteretic models was taken into account by considering the bilinear model and the stiffness degrading Q-model. Finally, in order to analyse the effect of damping, two models were considered: “mass-proportional” damping, which assumes a time-independent damping coefficient based on elastic properties, and “instantaneous stiffness-proportional” damping, which assumes a time-dependent damping coefficient based on tangent stiffness. For most cases, a damping ratio of 5\% was assumed, although for some systems a value of 2\% was adopted.\\

It is possible to derive approximate strength and displacement inelastic spectra from an elastic pseudo-acceleration spectrum using the proposed modified spectra. In the medium and long-period region, it was observed that the reduction factor is slightly dependant on the period T and is roughly equal to the prescribed ductility ($\mu$). However, in the short-period region, the factor C strongly depends on both T and $\mu$. The influence of hysteretic behaviour and damping can be observed for the whole range of periods. Based on this, a bilinear curve was proposed. Starting in C = 1, the value of C increases linearly along the short-period region up to a value approximately equal to the ductility factor. In the medium- and long-period range, the C-factor remains constant. This is mathematically expressed by the following relationships:


\begin{equation}
C_\mu = \left\{
\begin{matrix}
  c_{1}\left(\mu-1\right)^{c_{R}}\frac{T}{T_{0}} + 1, & T\leq T_{0}  \\
  c_{1}\left(\mu-1\right)^{c_{R}} + 1 & T>T_{0}
 \end{matrix}
 \right.
\end{equation} 

where: 
\begin{equation}
T_0 = c_2\mu^{c_T}T_c
\end{equation} 

And $T_c$ stands for the characteristic spectral period and $c_1$, $c_2$, $c_R$, $c_T$ are constants dependant on the hysteretic behaviour and damping model, as defined in Table \ref{table:VidicEtAl}.

\begin {table}[htb]
\caption{Paramereters for the estimation of the reduction factor C proposed by \citep{VidicEtAl1994}} 
\label{table:VidicEtAl} 
\begin{center}
  \begin{tabular}{ | c | c | c | c | c | c |}
    \hline
    Hysteresis model & Damping model & $c_1$ & $c_2$ & $c_R$ & $c_T$ \\ \hline
    Q & Mass & 1.00 & 0.65 & 1.00 & 0.30 \\ \hline
    Q & Stiffness & 0.75 & 0.65 & 1.00 & 0.30 \\ \hline
    Bilinear & Mass & 1.35 & 0.75 & 0.95 & 0.20 \\ \hline
    Bilinear & Stiffness & 1.10 & 0.75 & 0.95 & 0.20 \\ \hline
  \end{tabular}
\end{center}
\end{table}

In order to use this methodology, it is necessary to load one or multiple capacity curves as described in Section \ref{subsec:cap_curves}, as well as a set of ground motion records as explained in Section \ref{subsec:gmrs}. Then, it is necessary to specify a damage model using the parameter \verb=damage_model= (see Section \ref{subsec:dmg_model}), and a damping ratio using the parameter \verb=damping=. It is also necessary to specify the type of hysteresis (\verb=Q= or \verb=bilinear=) and damping (\verb=mass= or \verb=stiffness=) models as defined in Table \ref{table:VidicEtAl}, using the parameters \verb=hysteresis_model= and \verb=damping_model=, respectively. After importing the module \verb=vidic_etal_1994=, it is possible to calculate the distribution of structures across the set of damage state for each ground motion record using the following command:

\begin{Verbatim}[frame=single, commandchars=\\\{\}, samepage=true]
PDM, Sds = vidic_etal_1994.calculate_fragility(capacity_curves,gmrs,...
damage_model,damping)
\end{Verbatim}

Where \verb=PDM= (i.e. probability damage matrix) represents a matrix with the number of structures in each damage state per ground motion record, and \verb=Sds= (i.e. spectral displacements) represents a matrix with the maximum displacement (of the equivalent SDOF) of each structure per ground motion record. the variable PDM can then be used to calculate the mean fragility model as described in Section \ref{subsec:derive_fragility}.



