The nonlinear static procedures described in this section allow the calculation of the seismic response of a number of structures (in terms of maximum displacement of the equivalent single degree of freedom (SDOF) system), considering a set of ground motion records (see Section \ref{subsec:gmrs}). The development of this methods involved numerical analysis of systems with particular structural and dynamic properties (e.g. periods of vibration, viscous damping, hysteretic behaviour, amongst others) and accelerograms selected for specific regions in the world (e.g. California, South Europe). For these reasons, their applicability to other types of structures and considering a different ground motion records calls for due care. This section provides a brief description of each methodology, but users are adviced to fully comprehend the chosen methodology by reading the original publication. \\

The main results of each of these methodologies is a probability damage matrix (i.e. number of assets per damage state for each ground motion record, represented by the variable \verb=PDM=), and the spectral displacement (i.e. expected maximum displacement of the equivalent SDOF system, represented by the variable \verb=Sds=) per ground motion record. Using the probability damage matrix (\verb=PDM=), it is possible to derive a fragility model (i.e. probability of exceedance a number of damage states for a set of intensity measure levels - see Section \ref{subsec:derive_fragility}), or directly a vulnerability model (i.e. distribution of loss ratio for a set of intensity measure levels - see Section \ref{sec:derive_vulnerability}).