A consequence model (also known as damage-to-loss model), establishes the relation between physical damage and a measure of fraction of loss (i.e. the ratio between repair cost and replacement cost for each damage state). These models can be used to convert a fragility model (see Section \ref{subsec:derive_fragility}) into a vulnerability function (see Section \ref{subsec:derive_fragility}). \\

Several consequence models can be found in the literature for countries such as Greece \citep{KapposEtAl2006}, Turkey \citep{BalEtAl2010}, Italy \citep{DiPasqualeandGoretti2001} or the United States \citep{FEMA2003}. The damage scales used by these models may vary considerably, and thus it is necessary to ensure compatibility with the fragility model. Consequence models are also one of the most important sources of variability, since the economical loss (or repair cost) of a group of structures within the same damage state (say moderate) can vary significantly. Thus, it is important to model this component in a probabilistic manner.\\

In the Risk Modeller's Toolkit, this model is being stored in a \verb=csv= file (tabular format), as illustrated in Table \ref{table:cons_model}. In the first column, the list of the damage states should be provided. Since the distribution of loss ratio per damage state can be modelled using a probabilistic model, the second column must be used to specify which statistical distribution should be used. Currently, \verb=normal=, \verb=lognormal= and \verb=gamma= distributions are supported. The mean and associated coefficient of variation (cov) for each damage state must be specified on the third and fourth columns, respectively. Finally, each distribution should be truncated, in order to ensure consistency during the sampling process (e.g. avoid negative loss ratios in case a \verb=normal= distribution is used, or values above 1). This variability can also be neglected, by setting the coefficient of variation (cov) to zero.

\begin {table}[htb]
\caption{Example of a consequence model.}
\label{table:cons_model}
\begin{center}
  \begin{tabular}{ | c | c | c | c | c | c |}
  \hline
Damage States & distribution & Mean & Cov & A & B\\ \hline
Slight & normal & 0.1 & 0.2 & 0 & 0.2\\ \hline
Moderate & normal & 0.3 & 0.1 & 0.2 & 0.4\\ \hline
Extensive & normal & 0.6 & 0.1 & 0.4 & 0.8\\ \hline
Collapse & normal & 1 & 0 & 0.8 & 1\\ \hline
  \end{tabular}
\end{center}
\end{table}

