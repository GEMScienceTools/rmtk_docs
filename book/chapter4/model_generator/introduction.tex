The methodologies currently implemented in the Risk Modeller's Toolkit require the definition of the capacity of the structure (or building class) using a capacity curve (or pushover curve). These curves can be derived using software for structural analysis (e.g. SeismoStruct, OpenSees); experimental tests in laboratories; observation of damage from previous earthquakes; and analytical methods. The Risk Modeller's Toolkit provides two simplified methodologies (DBELA - \cite{SilvaEtAl2013}; SP-BELA - \cite{BorziEtAl2008b}) to generate capacity curves, based on the geometrical and material properties of the building class (thus allowing the propagation of the building to building vartiability). Moreover, it also features a module to generate sets of capacity curves, based on the median curve (believed to be representative of the building class) and the expected variability at specific points of the reference capacity curve.