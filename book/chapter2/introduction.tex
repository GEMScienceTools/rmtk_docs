The OpenQuake-engine is capable of generating several seismic hazard and risk outputs, such as loss exceedance curves, seismic hazard curves, loss and hazard maps, damage statistics, amongst others. Most of these outputs are stored using the Natural Risk Markup Language (nrml), or simple comma separated value (csv) files. The Plotting module of the Risk Modellers Toolkit allows users to visualize the majority of the OpenQuake-engine results, as well as to convert them into other formats compatible with GIS software (e.g. QGIS). Despite the default styling of the maps and curves defined within the Risk Modellers Toolkit, it is important to state that any user can adjust the features of each output by modifying the original scripts.