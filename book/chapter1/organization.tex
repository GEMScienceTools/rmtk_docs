This manual is designed to explain the various functions in the toolkit and to provide some illustrative examples showing how to implement them for particular contexts and applications. As previously indicated, the Risk Modeller’s Toolkit itself is primarily a Python library comprising three modules containing plotting tools, risk tools, and vulnerability tools respectively. The modular nature of the toolkit means that all of the functions within the toolkit can be utilised by a risk modeller in different python applications. In addition to the Python scripts, the RMTK also includes several interactive IPython notebooks illustrating sample usage of the functions in the toolkit. Each IPython notebook is intended to be stand-alone and self-explanatory and includes the most relevant information about the methodologies demonstrated in that notebook. In addition to the information provided in the notebooks, this manual provides more details about the theory and implementation, the algorithms and equations, the input model formats, and references for the different methodologies included in the toolkit.

Chapter~\ref{chap:introduction} describes the motivation behind the development of the RMTK, the installation and usage instructions, and the list of features that are currently implemented in the toolkit. Chapter~\ref{chap:plotting} describes the tools and notebooks provided in the plotting module of the RMTK for visualisation of OpenQuake hazard and risk results. Chapter~\ref{chap:risk} provides a brief description of the risk module, which includes two tools: (1) for postprocessing hazard curves from a classical PSHA calculation with non-trivial logic-trees and (2) for deriving probable maximum loss (PML) curves using loss tables from an event-based risk calculation. Chapter~\ref{chap:vulnerability} describes the different methodologies implemented in the vulnerability module of the RMTK for deriving seismic fragility and vulnerability functions for individual buildings or for a class of buildings. Several supplementary tools required in this process are also introduced and described in this chapter. Finally, Appendix~\ref{chap:python_guide} presents a brief introductory tutorial for the Python programming language.