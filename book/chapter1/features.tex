The Risk Modeller's Toolkit is currently divided into three modules:
\begin{description}
\item[Plotting tools:] The plotting module of the RMTK provides scripts and notebooks for visualising all of the different hazard and risk outputs produced by calculations performed using the OpenQuake-engine. The visualisation tools currently support plotting of seismic hazard curves, uniform hazard spectra, hazard maps, loss exceedance curves, probabilistic loss maps, and damage distribution statistics, amongst others. This module also allow users to convert the different results from the standard OpenQuake XML format into other formats such as CSV.
\item[Risk tools:] This module provides useful scripts to post-process OpenQuake hazard and risk results and calculate further metrics.
\item[Vulnerability tools:] This module implements several methodologies for estimating fragility and vulnerability functions for individual buildings or for a class of buildings. The provided scripts differ in level of complexity of the methodology and according to the type of input data available for the buildings under study. The guidelines for analytical vulnerability assessment provided by the GEM Global Vulnerability Consortium have been used as the primary reference for the development of this module.
\end{description}

A summary of the methods available in the present version is given in Table \ref{tab:current_features}.

\begin{table}[!htbp]
\centering
\begin{tabular}{|l|l|} \hline
\textbf{Module}     & \textbf{Feature / Methodology} \\ \hline
\textbf{Plotting}   & Hazard Curves \\
\textbf{Module}     & Uniform Hazard Spectra \\
                    & Hazard Maps \\
                    & Ground Motion Fields \\
                    & Collapse Maps \\
                    & Scenario Loss Maps \\
                    & Probabilistic Loss Maps \\
                    & Loss Curves \\
                    & Damage Distribution \\
                    & \\ \hline
\textbf{Risk}       & Probable Maximum Loss Calculator \\
\textbf{Module}     & Logic-Tree Branch Selector \\
                    & \\ \hline
\textbf{Vulnerability} & \textbf{Capacity Curve Generation} \\
\textbf{Module}        & DBELA \\
                    & SP-BELA \\
                    & Point Dispersion \\
                    & \\
                    & \textbf{MDOF$\to$SDOF Conversion} \\
                    & Single Mode of Vibration \\
                    & Adaptive Approach \\
                    & \\
                    & \textbf{Direct Nonlinear Static Methods} \\
                    & SPO2IDA \citep{VamvatsikosCornell2005} \\
                    & \citet{DolsekFajfar2004} \\
                    & \citet{RuizGarciaMiranda2007} \\
                    & \\
                    & \textbf{Record Based Nonlinear Static Methods} \\
                    & \citet{VidicEtAl1994} \\
                    & \citet{LinMiranda2008} \\
                    & \citet{Miranda2000} for Firm Soils \\
                    & N2 \citep{CEN2005} \\
                    & Capacity Spectrum Method \citep{FEMA4402005} \\
                    & DBELA \citep{SilvaEtAl2013} \\
                    & \\
                    & \textbf{Nonlinear Time History Analysis} \\
                    & for SDOF Oscillators \\ \hline
\end{tabular}
\caption{Current features in the OpenQuake Risk Modeller's Toolkit}
\label{tab:current_features}

\end{table}