The goal of this book is to provide a comprehensive and transparent description of the methodologies adopted during the implementation of the risk component of the OpenQuake in the Risk Modeller's Toolkit (RMTK). The Risk Modeller's Toolkit (RMTK) is a software suite for creating the input models required for running seismic risk calculations using the OpenQuake-engine. The RMTK implements several state-of-the-art methods for deriving robust analytical seismic fragility and vulnerability functions for single structures or building classes. The RMTK also provides interactive tools for post-processing and visualising different results from OpenQuake seismic risk calculations, such as loss exceedance curves, collapse maps, damage distribution, and loss maps.

The OpenQuake Risk Modeller's Toolkit is the result of an effort carried out jointly by the Information Technology and Scientific teams working at the Global Earthquake Model (GEM) Secretariat. It is freely distributed under an Affero GPL license (more information available at this link http://www.gnu.org/licenses/agpl- 3.0.html).