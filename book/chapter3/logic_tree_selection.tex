When a non-trivial logic-tree is used to capture the epistemic uncertainty in the source model, or in the choice of ground motion prediction equations (GMPE) for each of the tectonic region types of the region considered, the OpenQuake-engine can calculate hazard curves for each end-branch of the logic-tree individually. 

Should a risk modeller wish to just estimate the mean damage or losses of each asset in their exposure model, then they will only need the mean hazard curve. However, if they are interested in aggregating the losses from each asset in the portfolio, they should be using a Probabilistic Event-based Risk Calculator that makes use of spatially correlated ground motion fields per event, rather than hazard curves. For computational efficiency, it is useful to identify a branch of the logic tree that produces the aforementioned hazard outputs that are close to the mean, and this can be done by computing and comparing the hazard curves of each branch. Depending upon the distance of the hazard curve for a particular branch from the mean hazard curve, the risk modeller may can choose the branches for which the hazard curves are closest to the mean hazard curve. This Python script and corresponding IPython notebook allow the risk modeller to list the end-branches for the hazard calculation, sorted in increasing order of the distance of the branch hazard curve from the mean hazard curve. Currently, the distance metric used for performing the sorting is the root mean square distance.