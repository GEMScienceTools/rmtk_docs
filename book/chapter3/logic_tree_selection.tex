When a non-trivial logic-tree is used to capture the model uncertainty in the source model or in the choice of an appropriate ground motion prediction equation (GMPE) for each of the tectonic region types in the region considered, OpenQuake can calculate the hazard curves for each end-branch of the logic-tree individually. Now, if a risk modeller wishes to estimate damage or losses using one or a few of these branches only, it is useful to compare the hazard curves for the chosen branch with the mean hazard curve. Depending upon the distance of the hazard curve for a particular branch from the mean hazard curve, the risk modeller may wish to choose the branches for which the hazard curves are closest to the mean hazard curve. This Python script and corresponding IPython notebook allow the risk modeller to list the end-branches for the hazard calculation, sorted in increasing order of the distance of the branch hazard curve from the mean hazard curve. Currently, the distance metric used for performing the sorting is the root mean square distance.